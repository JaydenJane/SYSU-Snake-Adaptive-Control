\section{Relative work}
Currently, a widely used of control strategy for snake-like robots is based on the sinusoidal motion model\cite{HiroseSine} which is proposed by professor Hirose. After that, Tesch et al. proposed a parametric equation based on the sinusoidal model for a snake-shaped robot with three-dimensional athleticism\cite{ChosetSine}. This parametric equation simplifies the control strategy of the snake-like robots and allows the robot to determine the movement model of the machine with a small amount of control parameters.

The sinusoidal motion model function is shown in the following formula:
\begin{eqnarray}\label{basicRoll}
T_i=\left\{
\begin{array}{lr}
A\cdot \sin (\omega \cdot t + i\cdot \varepsilon )&odd\\
A\cdot \sin (\omega \cdot t + i\cdot \varepsilon +  \frac{\pi}{2})&even
\end{array}
\right.
\end{eqnarray}

By modifying the amplitude $A$, phase $\varepsilon$, and angular rate $\omega$ in Eq.\ref{basicRoll}, the maximum rotation angle of joints, the robot shape period and the motion rate of the serpentine robot are changed. In order to ensure that snake-like robots can be applied to a wider scene, robots need to have autonomous adaptability to the environment.

To adapt to the environment, the method which use the sensors to perceive the environment and embed the environment perception rules, has been widely used\cite{CPGenabling}\cite{GaitBasedCompliant}\cite{BalancingAndControl}\cite{FeedbackControlOfSoft}. Tang et al. proposed a control strategy based on CPG(central pattern generator) model\cite{CPGenabling}. They achieve to control the gait change to adapt to the environment by taking speed as a measure,  embedding several gaits to the robots and combining with CPG control according to the environment. Rollinson et al. proposed a snake-like robot adaptive control based on state estimation\cite{GaitBasedCompliant}. A complete correlation prediction model of control parameters is not given in their paper. As it is based on state estimation, this method is a step-by-step change model, thus it is not suited to the mutation environment. %There are some algorithms of machine learning in the robot control applications\cite{InformationDriven}\cite{NovelPlasticityRule}\cite{MissileSystems}\cite{NeuroFuzzyBayesian}. They only give control program conversion but do not provide changes in the control strategy.

On the application of machine learning in the field of robot control, there are researchers that have proposed a neural network model combined with physical environment information to determine the control scheme\cite{InformationDriven}\cite{NovelPlasticityRule}\cite{MissileSystems}\cite{NeuroFuzzyBayesian}. This model is only the control suggestion and may not make a good effect on motion when in real-time motion system of the robot. In this paper, a control strategy by experience-based learning  is proposed. Combined with the clustering and the multiple regression, we realize the real-time autonomous change of multiple control parameters in the robots' movement.
